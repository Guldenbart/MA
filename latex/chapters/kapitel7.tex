% -------------------------------------------------------------------
% Masterarbeit
% Kapitel 7: Fazit
% Autor: Daniel Fritz
% Datum: 04.05.2016
% -------------------------------------------------------------------

\chapter{Fazit und Ausblick}\label{chp:7:fazit}
In dieser Masterarbeit sollten folgende Ziele erreicht werden:

Ziel dieser Masterarbeit war die teilweise automatische Implementierung interner Java-DSLs. Grundlage einer Sprache sind dabei Interfaces, welche die Grammatik der Sprache definieren. Es sollte eine möglichst strikte Trennung der Komponenten der Implementierung untereinander erreicht werden. Verbunden damit war die Frage, ob diese (Generierung) den Entwicklungsprozess einer DSL verkürzen bzw. vereinfachen kann.

In den folgenden Abschnitten werden die Ergebnisse dieser Arbeit zusammengefasst. Zudem wird ein Ausblick auf künftige Entwicklungen gegeben.

\section{Ergebnisse}\label{sct:7.1:ergebnisse}
Die erste Phase dieser Abschlussarbeit bestand aus der manuellen Implementierung einer internen Java-DSL anhand einer einfachen Sprache für arithmetische Ausdrücke. In mehreren Iterations-Schritten wurden den Anforderungen genügende Daten-Strukturen entwickelt, die in Kapitel \ref{chp:4:implementierungstechniken} beschrieben sind.
Dabei spielte die Unabhängigkeit der Definition einer Sprache von ihrer Implementierung und die klare Trennung der Komponenten untereinander eine wichtige Rolle.

In der zweiten Phase 

\section{Ausblick}\label{sct:7.2:ausblick}